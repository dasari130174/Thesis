
\thesistitlepage % make the ordinary titlepage
\chapter*{Abstract}

Investigative Journalists have been playing a pivotal role in unearthing lot of hidden truths buried under the carpets. They have been instrumental in exposing incompetence, corruption, lies, abuse of power and broken promises etc. In the process of communicating with sources and collecting, analyzing, investigation, disseminating of information, the Interactive tools play a role of catalyst in performance of their job. J.D.Tygar and A Whitten most renowned researchers have identified a “weakest link property". The property states that  stating that muckrakers can exploit when users make error while using the digital tools. 

Therefore a single error made by investigative journalists while using digital tools may lead to loose data and there will be a chance for muckrakers to find out their sources identity. This research is oriented towards the study, evaluation and improving usability of security tools by considering mental model of the investigative journalists. There are many digital security tools currently available for investigative journalists like Tor browser, Text Secure, Red Text and so on for protecting their data and sources. But the tools mentioned above faces usability issues by which users may tend to stop using the tool gradually or may tend to make errors. Most of the security tools available in the market are developed for larger audience and not specifically for investigative journalists and according to Pew research centre many investigative journalists don’t use digital security tools to protect their digital privacy because there more chance for loosing the data and they can be traced by geo-location and so on.

Therefore, in this thesis I will design an application by considering the investigative journalist requirements. Furthermore, the tool will be designed in such a way that both investigative journalists who are technically sound and those are who aren’t; are comfortable in operating/using is also emphasised. Improvisation of usability tools and enabling all the Investigative Journalists to enjoy the fruits of technological breakthroughs, emerging out of day-to-day research work by consolidation and orientation towards employing cognitive walkthroughs, interviews, usability testing.


%\include{summary}

\chapter*{Preface}

I would like to thank Cristina


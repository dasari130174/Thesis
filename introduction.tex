\chapter{Introduction}
\label{chap:introduction}

It is broadly acknwoledged that security tools are only effective when they are used by all the end users, who are technically savvy and those who aren't. Technically well implemented code will not provide security if users are unable to use the primary features of the tool. Human errors are one of the main reason for most of the security system failures. However, still the user interfaces or the features of the security tool are designed to be clumsy and confusing. Therefore, this research will focus on the need to improve the usability for security tools which are used by investigative journalists like Tor, Cryptocat, TextSecure and so on ~\cite{wm4}. The reasons for the security system failures are assessed by evaluating the existing tools.
\section{Keywords}
Privacy, anonymity, investigative journalist, usability. 

\section{Problem description}\label{Problem:description}
"There is a usability gap that is transformed directly into usage gap", if users don't know how to use the security tool even with appropriate awareness~\cite{furnell2006challenges}. However, this gap may force the end user to stop using the tool or to mis-use the key security features like encryption, public keys, private keys, signatures and so on. Generally, security tools provide privacy and anonymity for the end-users and also protects information from their adversaries. Moreover, privacy and anonymity depends on the set of users using the tools. Larger the set, higher the anonymity and privacy are achieved~\cite{dingledine2006anonymity}. For suppose if the number of users stops using the security tool which provides anonymity and privacy, then privacy levels get decreased for the existing user.  In order to increase  privacy and anonymity levels for both  existing user and new user. The tool should be designed in a way that it can be used by users, who are technologically savvy and also by those who are not. Then, there may be a chance for more numbers users to use the security tool~\cite{pfitzmann2001anonymity}. Therefore, usability affects the security in systems that aim to protect data confidentiality. But when the goal is privacy, it becomes even more important ~\cite{dingledine2006anonymity}.

There are some secured tools like Tor Browser, RedPhone, TextSecure, Cryptocat and so on, which are highly recommended for investigative journalists to protect the data and identity of the sources~\cite{wm4}. Tor browser avoids someone viewing your web history and provides access to sites that are blocked \footnote{https://www.torproject.org}. RedPhone and TextSecure has the capability to secure the exchange of conversations made from their personal mobile phone/smart phones and provide end to end encryption. Chat and Text messages are safe and cannot be decoded even if the phone is lost \footnote{https://whispersystems.org}. Cryptocat provides encrypted chat in browser and mobile phone\footnote{https://crypto.cat}. These tools are technically well implemented, but they are haunted by usability issues like a frequent crash of web pages, missing translations and extraction issues for Tor browser~\cite{pa3}. Connectivity issue within the interfaces due to various configuration for different devices and issues like no feedback to the user is provided when the problem is raised for RedPhone~\cite{sr4}. Duplication of contacts in the contact list and some issues with the color interface design  and feedback to the user for TextSecure~\cite{ti4}. Usability issues due to language differences and lack of cultural integration for Cryptocat~\cite{kobeissi2013cryptocat}.

Investigative journalists spend months or years to verify in-depth truths, discover hidden secrets, shed a spotlight on social justice and accountability, that may involve crime, corruption and corporate wrongdoing etc. During investigating or researching on any social issue, international or any other issues, investigative journalists contact their source and colleagues, located across the globe.~\cite{Ijnet14} They are highly protective about their data and  contacts because adversaries may try to mislead the investigations or threaten investigative journalists; their colleagues and sources. The usability issues that are described above may prevent the investigative journalists to use the features that are essential. These issues may also lead to misuse of features with out the knowledge of investigative journalists. Which results in creating a risk for them, their sources, data and colleagues. At present, It is broadly acknowledged that security requirements can't be addressed by technical means alone, and the chances of success will be essentially affected by the users involved ~\cite{furnell2006challenges}. Therefore, this research will coherently evaluate current existing tools that are used by investigative journalists, explore the challenges of usable security, prevailing problems and propose a usable interactive solution based on observations. 
\section{Justification, motivation and benefits}
The assessment and improvement of security tools in terms of usability are  important for investigative journalists to maintain high privacy and anonymity levels. But if improved, it  helps in minimize the human and interface errors that would occur while using the security tools~\cite{furnell2006challenges}. Chances of risk get decearsed and provides support for investigative journalists to discover hidden truths and perform investigations with regard to social justice.

Many of the previous works have expressed and showed how important it is to improve usability in secure system tools, like in the work done by J.D.Tygar et al ~\cite{whitten1999johnny}. Evaluating the security tool PGP 5.0 to know whether is it usable and produce effective security of most computer users. After evaluating the PGP 5.0 tool, they have observed that   desipte of it's attractive computer interface, it is not usable for most its computer users~\cite{whitten1999johnny}.

The identified problem when solved can be very beneficial for investigative journalists during situations like discovery of truths about government officials (who otherwise are powerful people and any slackness in maintaining secrecy at investigation stage will be hazardous and  may obliterate the whole exercise), investigating about a bank robbery or about any issue when they are accountable for public etc. The soluble will help the investigative journalists to maintain confidentiality of the information and identity of the source from their adversaries.


\section{Research questions}\label{research:questions}
The research questions included in this work are:
\begin{enumerate}
	
	\item How are the existing secured tools used by investigative journalists are user friendly?
	\item How does the existing tools provide a dependable amount of privacy and anonymity while maintaining usability?
	\item If privacy and anonymity are comprised due to design flaws in terms of usability, then how can it be fixed?
	
	
\end{enumerate}

\section{Planned contributions}
In this research the problem addressed above can be solved by considering the previous best resulted works~\cite{whitten1999johnny}~\cite{clark2007usability}. In this work a more structured and precise steps will be implemented by considering most useful factors to find a better solution to the problem. The work from ~\cite{whitten1999johnny}~\cite{clark2007usability}~\cite{furnell2006challenges}~\cite{dingledine2006anonymity}~\cite{kobeissi2013cryptocat}~\cite{katsabas2005using} are extracted and organized to formulate a new methodology to provide a more improved solution to the problem addressed.


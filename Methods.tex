\chapter{Methods}
 Research methods are classified in to two types 1) Qualitative Method and 2)Quantitative Method ~\cite{lazar2010research}. In our research, we are implementing the Qualitative method. To address the first and second research question, we will perform a case study on  three existing tools described in ~\ref{Problem:description}: Cryptocat, TextSecure and RedPhone. The case studies in human computer interaction(HCI) have four goals:~\cite{lazar2010research}
 \begin{enumerate} 
 	\item \textbf{Exploration}: Understanding the problems or situations, often with the hopes of informing new design.
 	\item \textbf{Explanation}: Developing models that can be used to understand a context of technology.
 	\item \textbf{Description}: documenting a system, a context of technology use, or the process that lead to proposed design.
 	\item \textbf{Demonstration}: Showing how a new tool is sucessfully used.~\cite{lazar2010research}
 \end{enumerate}
 Therefore, the case studies helps us in understanding the problems faced by investigatice journalists while using the digital security tools and it will also  help in developing a process based on the results. There are four components of a case study design: research questions; hypothesis or propositions; units of analysis;data analysis plan. Research questions describes the goals of thesis and hypothesis or propositions are the statemetns of what you expect to find. The unit of analysis determines the focus group and finally data analysis plan is helpful in planning the data collection.~\cite{lazar2010research} The research questions are mentioned above in section ~\ref{research:questions}and~\ref{Problem:description}. The unit of analysis is a group of investigative journalists and in this research work the data analysis plan start firstly by  employing cognitive walkthrough method to  evaluate the digital security tools mention in section ~\ref{Problem:description}, as it has been used previously by the researchers mentioned in section~\ref{chap:relatedwork}. Cognitive walkthroughs are often very good at identifying certain classes of problems, especially showing how easy or difficult a system is to learn or explore effectively – how difficult it will be to start using that system without reading the documentation, and how many false moves will be made in the meantime.~\cite{wharton1994cognitive}\\
 
 
 hieeieieieie
 
 In cognitive walkthrough, set of core tasks are developed and these tasks are designated differently for the three digital security tool mention above. We will perform these tasks and evaluate the usability of three security tools: Cryptocat, TextSecure and RedPhone against set of guidelines. These guide lines are to be derived from the previous works~\cite{whitten1999johnny}~\cite{clark2007usability}~\cite{katsabas2005using}~\cite{furnell2006challenges}. To address the third reseach question, although the results from the cognitive walkthrough may provide valuable information regard to the problems faced in existing tools. In this research we will also use contextual inquiry to draw the user requirements. It is one of the best methods to understand the users' work context and it is basically a structured field interviewing method.\\
 
 Contextual inquiry is more a discovery process than an evaluative process; more like learning than testing. This technique is best used in the early stages of development, to gain and understand  how people feel about their jobs; how they carry out their work; how information flows through the organisation, etc.~\cite{1_holtzblatt_2014} According to ~\cite{4nielsen2014} stated that ``The best results come from testing no more than 5 users" because at the best they provide better results. Therefore, contextual inquiry for this research work, we will test no more than 5 users.   Based on the results from cognitive walkthrough and contextual inquiry, we will develope a paper portotype. \\
 
 After developing a paper prototype, we will use the method card sorting,  it helps in understand the investigative journalists expectations. Card sorting is a method The card sorting helps in designing the information architecture according to user needs. By conducting card sorting, there is a chance for getting more insights from the users. 
 ~\cite{2_hudson_2014} Then by considering results that are dervied from card sorting, cognitive walkthroughs and contextual enquiry, we will design the high fidelity prototype. Prototype which is close to the final product. Then perform a usablity testing for evaluating high fidelity prototype, as it produces more objective results. A usability test is conducted to draw the usability problems and this method is conducted mostly in earlier stages or the final stages of the designing process. For conducting the usablity test we will design a real test scenario and ask the users to use the prototype in realtion to scenario. 